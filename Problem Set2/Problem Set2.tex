\documentclass[12pt]{article}
\usepackage{fullpage}
\usepackage{graphicx}
\usepackage{hyperref}
\usepackage{breakurl}
\usepackage{bm}
\usepackage{amsmath}
\usepackage{amssymb}
\usepackage{derivative}
\usepackage{bm}

\begin{document}
\title{Mathematical Proof: Problem Set 2}
\author{Koichiro Takahashi}
\maketitle 

\section*{Problem.1}
Here, $U = \{1,2,3\}, A = \{1,2\}, B = \{2,3\}$ and $C = \{1,3\}$.\\[1em]
$(a)$
\begin{align*}
(A \cup B) - (B \cap C) &= (\{1,2\} \cup \{2,3\}) - (\{2,3\} \cap \{1, 3\})\\[1em]
&= \{1,2,3\} - \{3\} \\[1em]
&= \{1,2\}
\end{align*}
$(b)$
\begin{align*}
\overline{A} = U - A = \{1,2,3\} - \{1,2\} = \{3\}
\end{align*}
$(c)$
\begin{align*}
\overline{B \cup C} &= U - (B \cup C)\\[1em]
&= \{1,2,3\} - (\{2,3\} \cup \{1,3\})\\[1em]
&= \{1,2,3\} - \{1,2,3\} \\[1em]
&= \emptyset \\[1em]
\end{align*}
$(d)$
\begin{align*}
A \times B &= \{1,2\} \times \{2,3\}\\[1em]
&= \{(1,2),(1,3),(2,2),(2,3)\}
\end{align*}
$(e)$
By using the result in $(b)$,
\begin{align*}
\overline{A} \times C &= \{3\} \times \{1,3\}\\[1em]
&= \{(3,1),(3,3)\}
\end{align*}

\section*{Problem.2}
Here, $A = \{1\}$ and $C = \{1,2\}$.\\[1em]
\begin{gather*}
\mathcal{P}(A) = \{\emptyset, \{1\}\}\\[1em]
\mathcal{P}(C) = \{\emptyset, \{1\}, \{2\}, \{1,2\}\}
\end{gather*}
$\therefore$ an example of a set $B~s.t.~\mathcal{P}(A) \subset B \subset \mathcal{P}(C)$ is 
\begin{gather*}
B = \{\emptyset, \{1\}, \{1,2\}\}
\end{gather*}

\section*{Problem.3}
Here, $R = \{(1,1),(1,2),(1,3),(2,1),(4,2),(4,3)\}$.\\[1em]
Now, to find the subsets $A, B, C, D \subset \{1, 2, 3, 4\}~s.t.~ R = ((A \times B) \cup (C \times D)) - (D \times D)$, first one can think about $D$.
In $R$, since only $(1,1)$ is the ordered pair by the same element, we take $D = \{2, 3\}$. Then we should take $C = \{1, 4\}$ to generate the appropriate ordered pairs in $R$. Considering the rest elements in $R$ that cannot be generated by $C$ and $D$, you take $A = \{1,2\}, B = \{1\}$.
Therefore, if you take the subsets $A, B, C, D \subset \{1, 2, 3, 4\}$ as
\begin{gather*}
A = \{1,2\},~B = \{1\},~C = \{1, 4\},~D = \{2, 3\},
\end{gather*}
certainly this can be shown to satisfy the condition as follows.
\begin{align*}
&((A \times B) \cup (C \times D)) - (D \times D)\\[1em]
&= ((\{1,2\} \times \{1\}) \cup (\{1, 4\} \times \{2, 3\})) - (\{2, 3\} \times \{2, 3\})\\[1em]
&= (\{(1,1),(2,1)\} \cup \{(1,2),(1,3),(4,2),(4,3)\}) - \{(2,2),(2,3),(3,2),(3,3)\}\\[1em]
&= \{(1,1),(1,2),(1,3),(2,1),(4,2),(4,3)\} - \{(2,2),(2,3),(3,2),(3,3)\}\\[1em]
&= \{(1,1),(1,2),(1,3),(2,1),(4,2),(4,3)\} = R
\end{align*}
※ The impossibility of finding the subsets\\[1em]
\begin{gather*}
A, B, C, D \subset \{1, 2, 3, 4\}~s.t.~ R = ((A \times B) \cup (C \times D)) - (D \times D),
\end{gather*}
where
\begin{gather*}
R = \{(1,1),(2,1),(3,2),(3,3),(4,2),(4,3)\}
\end{gather*}

\section*{Problem.4}
Here, the statements;\\[1em]
$P$: I pass the test.\\[1em]
$Q$: I pass the course.\\[1em]
$R$: I make the dean's list.\\[1em]
(a)(i)
\begin{center}
$P \lor (\sim Q)$: I pass the test or don't pass the course.
\end{center}
(ii)
\begin{center}
$(\sim P) \Rightarrow (\sim Q)$: If I don't pass the test, then I don't pass the course.
\end{center}
(iii)
\begin{center}
$Q \Rightarrow P$: If I pass the course, then I pass the test.
\end{center}
(iv)
\begin{center}
$(\sim P) \Rightarrow (\sim R)$: If I don't pass the test, then I don't make the dean's list.
\end{center}
(v)
\begin{center}
$(\sim P) \land Q$: I don't pass the test and pass the course.
\end{center}
(vi)
\begin{center}
$(P \Rightarrow Q) \land (Q \Rightarrow R)$: If I pass the test, then I pass the course, and if I pass the course, then I make the dean's list.
\end{center}
(vii)
\begin{center}
$Q \land (\sim R)$: I pass the course and don't make the dean's list.
\end{center}
(viii)
\begin{center}
$Q \Rightarrow R$: If I pass the course, then I make the dean's list.
\end{center}
(ix)
\begin{center}
$(P \Rightarrow R) \land (R \Rightarrow P) \equiv (R \Leftrightarrow P)$: I pass the test if and only if I make the dean's list.
\end{center}
(b)(i)
\begin{gather*}
P \Rightarrow R
\end{gather*}
(ii)
\begin{gather*}
Q \lor (\sim R)
\end{gather*}
(iii)
\begin{gather*}
(\sim Q) \Rightarrow (\sim R)
\end{gather*}
(iv)
\begin{gather*}
P \Leftarrow Q
\end{gather*}
(v)
\begin{gather*}
Q \land (\sim R)
\end{gather*}
(vi)
\begin{gather*}
P \Leftrightarrow Q
\end{gather*}
\section*{Problem.5}
(a)
\begin{displaymath}
\begin{array}{|c|c|c|}
P & Q & P \land Q\\ 
\hline
T & T & T\\
T & F & F\\
F & T & F\\
F & F & F\\
\end{array}
\end{displaymath}
(b)
\begin{displaymath}
\begin{array}{|c|c|c|c|}
P & Q & P \Rightarrow Q & (P \Rightarrow Q) \lor P\\ 
\hline
T & T & T & T\\
T & F & F & T\\
F & T & T & T\\
F & F & T & T\\
\end{array}
\end{displaymath}
(c)
\begin{displaymath}
\begin{array}{|c|c|c|c|c|}
P & Q & \sim P & \sim Q & (\sim P) \lor (\sim Q)\\ 
\hline
T & T & F & F & F\\
T & F & F & T & T\\
F & T & T & F & T\\
F & F & T & T & T\\
\end{array}
\end{displaymath}
(d)
\begin{displaymath}
\begin{array}{|c|c|c|c|c|}
P & Q & P \Rightarrow Q & \sim P & (P \Rightarrow Q) \Rightarrow (\sim P)\\ 
\hline
T & T & T & F & F\\
T & F & F & F & T\\
F & T & T & T & T\\
F & F & T & T & T\\
\end{array}
\end{displaymath}
(e)
\begin{displaymath}
\begin{array}{|c|c|c|c|c|}
P & Q & P \Rightarrow Q & \sim (P \Rightarrow Q) & P \Rightarrow (\sim (P \Rightarrow Q))\\ 
\hline
T & T & T & F & F\\
T & F & F & T & T\\
F & T & T & F & T\\
F & F & T & F & T\\
\end{array}
\end{displaymath}

(※(d)\&(e) is contraposition)
\section*{Problem.6}
Here, $A = \{4,5,6,7,8,9\}$, and the open sentences\\[1em]
\begin{gather*}
P(B): B \cap \{4,6,8\} = \emptyset\\[1em]
Q(B): B \neq \emptyset
\end{gather*}
over the domain $\mathcal{P}(A)$.\\[1em]
(a) For $P(B) \land Q(B)$ to be true,
$B$ is nonempty and is disjoint with $\{4,6,8\}$.
Therefore, let $C$ be a set given by
\begin{align*}
C &= \mathcal{P}(A-\{4,6,8\}) - \{\emptyset\}\\[1em]
&= \mathcal{P}(\{5,7,9\}) - \{\emptyset\}\\[1em]
&= \{\{5\},\{7\},\{9\},\{5,7\},\{5,9\},\{7,9\},\{5,7,9\}\}
\end{align*}
and $\forall B \in C \subset \mathcal{P}(A)$ satisfy $P(B) \land Q(B)$ is true, and that is all.\\[1em]
(b) For $P(B) \lor (\sim Q(B))$ to be true,
$B$ has to be an empty set, or has to be disjoint with \{4,6,8\}.
So let $D$ be a set given by
\begin{align*}
D &= \mathcal{P}(A-\{4,6,8\})\\[1em]
&= \mathcal{P}(\{5,7,9\})\\[1em]
&= \{\emptyset,\{5\},\{7\},\{9\},\{5,7\},\{5,9\},\{7,9\},\{5,7,9\}\}
\end{align*}
and $\forall B \in D \subset \mathcal{P}(A)$ satisfy $P(B) \lor (\sim Q(B))$ is true, and that is all.\\[1em]
(c) For $(\sim P(B)) \land (\sim Q(B))$ to be true,
first the only element in $\mathcal{P}(A)$ satisfy that $\sim
~Q(B)$ is true is an empty set, but $(\sim P(\emptyset))$ is false. Therefore, $B \in \mathcal{P}(A)$ which satisfy $(\sim P(B)) \land (\sim Q(B))$ does not exist.
\section*{Problem.7}

Suppose the set $S$ cardinality of 2 exists, given by 
\begin{gather*}
S = \{a,b\}
\end{gather*}
where $a\neq b$,~$P(a)\Rightarrow Q(a),~Q(a)\Rightarrow R(a)$,~and $R(b) \Rightarrow P(b)$ are true, and also $P(a)\Leftarrow Q(a),~Q(a)\Leftarrow R(a)$,~and $R(b) \Leftarrow P(b)$ are false.~($\because$ one can always define $a,~b$~in this way without loss of generality, because the open sentences are cyclic, and over the domain, the pigeonhole principle of dividing three open sentences into two elements applies.)\\[1em]
In general, to be the implication $P(n)\Rightarrow Q(n)$ to be false, $P(n)$ is true and $Q(n)$ is false are necessary.\\
Thus, $a \in S$ satisfies that $Q(a) \Leftarrow R(a)$ is false, therefore $Q(a)$ is false.\\[1em]
However, this contradicts that $P(a) \Leftarrow Q(a)$ is false because of the law of false hypothesis.\\[1em]
Therefore, such a set $S$ does NOT exist.

\end{document}
























