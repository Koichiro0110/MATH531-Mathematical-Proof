\documentclass[12pt]{article}
\usepackage{fullpage}
\usepackage{graphicx}
\usepackage{hyperref}
\usepackage{breakurl}
\usepackage{bm}
\usepackage{amsmath}
\usepackage{amssymb}
\usepackage{derivative}
\usepackage{bm}

\begin{document}

\title{Mathematical Proof: Problem Set 1}

\author{Koichiro Takahashi}

% \maketitle creates the title, author list, and submission date
\maketitle 

\section*{Problem.1}
$(a)$ is NOT a set, since the curly brackets are not properly matched and does not satisfy the definition of a set.\\[1em]
$(b)$ is a set, since the curly brackets are properly matched and satisfies the definition of a set.\\[1em]
$(c)$ is a set, since the curly brackets are properly matched and satisfies the definition of a set.\\[1em]
$(d)$ is NOT a set, since the objects are not enclosed in curly brackets and does not satisfy the definition of a set.

\section*{Problem.2}
$(a)$
\begin{gather*}
A = \{-2, -1, 0, 1, 2, 3, 4, 5\}
\end{gather*}
$(b)$
\begin{gather*}
B = \{1, 4, 9, 16, 25, 36, 49, 64, 81\}
\end{gather*}
$(c)$
\begin{gather*}
C = \{-64, -27, -8, -1, 0, 1, 8, 27, 64\}
\end{gather*}
$(d)$
\begin{gather*}
D = \emptyset
\end{gather*}
$(e)$
\begin{gather*}
E = \{0,1\}
\end{gather*}

\section*{Problem.3}
$(a)$
\begin{gather*}
A = \{0\},~B = \{\{0\},1\},~C = \{\{\{0\},1\},2\}
\end{gather*}
$(b)$
\begin{gather*}
A = \{0\},~B = \{\{0\},1\},~C = \{0, 1\}
\end{gather*}
$(c)$
\begin{gather*}
A = \{0, 1\},~B = \{1, 2\},~C = \{1, \{0, 1\}, \{1, 2\}\}
\end{gather*}
\begin{gather*}
A \cap B = \{1\} \subset C,~A \cup B = \{0, 1, 2\} \notin C
\end{gather*}

\section*{Problem.4}
\begin{gather*}
A = \{-1, 0, 1\},~B = \{-1, 0, 1\},~C = \{0, 1\},~D = \{-1,0,1\},~E = \{-1,0,1\}
\end{gather*}
\begin{gather*}
\therefore A = B = D = E 
\end{gather*}

\section*{Problem.5}
Here, $ i \in \mathbb{Z}, B_i = \{i-1,i,i+1\} $. \\[1em]
$(a)$
\begin{align*}
\cap^{3}_{i=1} \left(B_{2i-1} \cup B_{2i+1}\right) &= \left(B_{1} \cup B_{3}\right) \cap \left(B_{3} \cup B_{5}\right) \cap \left(B_{5} \cup B_{7}\right)\\[1em]
&= \left(\{0,1,2\} \cup \{2,3,4\}\right) \cap \left(\{2,3,4\} \cup \{4,5,6\}\right) \cap \left(\{4,5,6\} \cup \{6,7,8\}\right)\\[1em]
&= \{0,1,2,3,4\} \cap \{2,3,4,5,6\} \cap \{4,5,6,7,8\}\\[1em]
&= \{4\}
\end{align*}


$(b)$
\begin{gather*}
B_{i} \cap B_{i+1} = \{i-1,i,i+1\} \cap \{i,i+1,i+2\} = \{i,i+1\}
\end{gather*}
\begin{align*}
\therefore \cup^{n}_{i=1} \left(B_{i} \cap B_{i+1}\right) &= \cup^{n}_{i=1} \{i,i+1\} = \{1,2\} \cup \{2,3\} \cup \dots \cup \{n,n+1\}\\[1em]
&= \{1,2,\dots,n,n+1\} = \{l \in \mathbb{N}~|~ 1 \leq l \leq n+1\}~~(n \in \mathbb{N})
\end{align*}

$(c)$
\begin{gather*}
B_{3i} \cup B_{-3i} = \{3i-1,3i,3i+1\} \cup \{-3i-1,-3i,-3i+1\} = \{-3i-1,-3i,-3i+1,3i-1,3i,3i+1\}
\end{gather*}
\begin{align*}
\therefore \cup^{\infty}_{i=0} \left(B_{3i} \cup B_{-3i}\right) &= \cup^{\infty}_{i=0} \{-3i-1,-3i,-3i+1,3i-1,3i,3i+1\}\\[1em]
&= \{-1,0,1\} \cup \{-4,-3,-2,2,3,4\} \cup \{-7,-6,-5,5,6,7\} \cup \dots\\[1em]
&= \mathbb{Z}\\[1em]
(&\because 3(i+1)-1 = 3i+2 > 3i + 1, -3(i+1)+1 = -3i-2 < -3i -1) 
\end{align*}

\section*{Problem.6}
Here, $A = \{1,2,3,4,\dots,10\}$.\\[1em]
$(a)$ A partition that meets the requirements of this problem $S$ is
\begin{gather*}
S = \{\{1\}, \{2,3,4\}, \{5,6,7,8,9,10\}\}.
\end{gather*}
$(b)$ Let $A$ a set, if no two cells in that partition have the same cardinality, the greatest number of cells that can be taken in such a partition $|S|_{max}$ is the number of elements of natural numbers added from 1 in order and become larger than $|A|$, minus one.
In this case, $|A| = 10$, and $1+2+3+4 = 10, 1+2+3+4+5 = 15 > 10$.
\begin{gather*}
\therefore |S|_{max} = 4
\end{gather*}
$(c)$ 
Given that the cells of a partition are nonempty and pairwise disjoint,
the smallest cardinality a set B can have and still possess a partition with exactly three cells, no two of which have the same cardinality $|B|_{min}$, is the number of natural numbers added from 1 to 3.
\begin{gather*}
\therefore |B|_{min} = 1 + 2 + 3 = 6
\end{gather*}

\section*{Problem.7}
Here, $r \in \mathbb{R},~r \geq 0,~C_r = \{(x,y) \in \mathbb{R} \times \mathbb{R}~|~x^2 + y^2 = r^2\}$, and $\mathcal{D} = \{C_r~|~r \in \mathbb{R} ~\mathrm{and}~ r > 0\}$.\\
If $r > 0 \Rightarrow r^2 > 0$, therefore
\begin{gather*}
0 < r^2 = x^2 + y^2 \Rightarrow x^2 > 0~\mathrm{or}~y^2 > 0 \Rightarrow x \neq 0~\mathrm{or}~y \neq 0.
\end{gather*}
Thus, even though $(0,0) \in \mathbb{R} \times \mathbb{R}$,
\begin{gather*}
 (0,0) \notin \cup_{r \in \mathbb{R}, r > 0}  C_r \neq \mathbb{R} \times \mathbb{R}
\end{gather*}
and $\mathcal{D}$ is NOT a partition of the Cartesian plane $\mathbb{R} \times \mathbb{R}$.
\end{document}
























