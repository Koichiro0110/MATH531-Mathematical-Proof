\documentclass[12pt]{article}
\usepackage{fullpage}
\usepackage{graphicx}
\usepackage{hyperref}
\usepackage{bm}
\usepackage{amsmath}
\usepackage{amssymb}
\usepackage{derivative}
\usepackage{bm}
\usepackage{comment}
\usepackage{cancel}
\usepackage{xcolor}


\begin{document}
\title{Mathematical Proof: Problem Set 4}
\author{Koichiro Takahashi}
\maketitle 
Recall $\mathbb{E}, \mathbb{O}$ are sets of even and odd integers respectively.\\
Also, we know the theorems from the textbook:
\begin{center}
Theorem 3.16:
Let $x,y \in \mathbb{Z}$. Then, $x$ and $y$ are of the same parity\\[1em]
if and only if $x+y$ is even.
\end{center}
\begin{center}
Theorem 3.17: Let $x,y \in \mathbb{Z}$. Then $xy$ is even if and only if $x$ is even or $y$ is even.
\end{center}
For the use in Problem.6, we introduce the corollaries of the Theorem 3.17:
\begin{center}
Corollary 1: Let $x \in \mathbb{Z}$. $x^2 \in \mathbb{O} \Rightarrow x \in \mathbb{O}$
\end{center}
\begin{center}
Corollary 2: Let $x \in \mathbb{Z}$. $x^2 \in \mathbb{E} \Rightarrow x \in \mathbb{E}$
\end{center}

\section*{Problem.1}
\underline{\textit{Proof}}: Let $x, y, z \in \mathbb{Z}$.\\[1em]
\underline{\textit{Case $1$}}: $x, y \in \mathbb{E},~z \in \mathbb{O}$.\\
Therefore, $3x \in \mathbb{E}$, $5y \in \mathbb{E}$, $7z \in \mathbb{O}$ ($\because$ Theorem 3.17). And immediately, $3x + 5y \in \mathbb{E}$, $3x + 5y + 7z \in \mathbb{O}$ ($\because$ Theorem 3.16).\\[1em]
\underline{\textit{Case $2$}}: $x, z \in \mathbb{E},~y \in \mathbb{O}$.\\
Therefore, $3x \in \mathbb{E}$, $5y \in \mathbb{O}$, $7z \in \mathbb{E}$ ($\because$ Theorem 3.17). And immediately, $3x + 5y \in \mathbb{O}$, $3x + 5y + 7z \in \mathbb{O}$ ($\because$ Theorem 3.16).\\[1em]
\underline{\textit{Case $3$}}: $y, z \in \mathbb{E},~x \in \mathbb{O}$.\\
Therefore, $3x \in \mathbb{O}$, $5y \in \mathbb{E}$, $7z \in \mathbb{E}$ ($\because$ Theorem 3.17). And immediately, $3x + 5y \in \mathbb{O}$, $3x + 5y + 7z \in \mathbb{O}$ ($\because$ Theorem 3.16).\\[1em]
From the above, the statement is true.~$\blacksquare$
\begin{comment}
By definition,
\begin{gather*}
\exists k, l, m \in \mathbb{Z},~s.t.~ x = 2k,~y = 2l,~z = 2m+1
\end{gather*}
Therefore,
\begin{gather*}
3x + 5y + 7z = 
\end{gather*}
\end{comment}

\section*{Problem.2}
\underline{\textit{Proof}}: Let $a, b \in \mathbb{Z}$. Suppose $ab = 4$.\\
\begin{align*}
(a-b)^3 - 9(a-b) &= (a-b)\{(a-b)^2-9\}\\[1em]
&= (a-b)(a^2 + b^2 -2ab -9)\\[1em]
&= (a-b)(a^2 + b^2 -17) ~~(\because ab = 4)
\end{align*}
Therefore,
\begin{align*}
(a-b)^3 - 9(a-b) = 0 \Leftrightarrow a-b = 0,~\mathrm{or},~ a^2 + b^2 -17 = 0
\end{align*}
\underline{\textit{Case $1$}}: $a = \pm 4,~b = \pm 1$ (double sign in the same order).\\
Then,
\begin{gather*}
a^2 + b^2 -17 = 16 + 1 - 17 = 0
\end{gather*}
\underline{\textit{Case $2$}}: $a = \pm 2, b = \pm 2$ (double sign in the same order).\\
Then,
\begin{gather*}
a - b = \pm (2 - 2) = 0
\end{gather*}
\underline{\textit{Case $3$}}: $a = \pm 1, b = \pm 4$ (double sign in the same order).\\
Then,
\begin{gather*}
a^2 + b^2 -17 = 1 + 16 - 17 = 0
\end{gather*}
From the above, the statement is true.~$\blacksquare$

\section*{Problem.3}
\underline{\textit{Proof}}: Let $a \in \mathbb{Z}$. 
We prove by contrapositive.\\
Suppose $3 \nmid a$. By definition,
\begin{gather*}
\exists k \in \mathbb{Z},~s.t.~ a = 3k + 1 ~\mathrm{or}~ 3a = k + 2
\end{gather*}
\underline{\textit{Case $1$}}: $a = 3k + 1$.
Then,
\begin{gather*}
2a = 2(3k + 1) = 3(2k) + 2
\end{gather*}
Since $2k \in \mathbb{Z}, ~3 \nmid 2a$.\\[1em]
\underline{\textit{Case $2$}}: $a = 3k + 2$.
Then,
\begin{gather*}
2a = 2(3k + 2) = 3(2k+1) + 1
\end{gather*}
Since $ 2k+1 \in \mathbb{Z}, ~3 \nmid 2a$.\\[1em]
From the above, the statement is true.~$\blacksquare$

\section*{Problem.4}
\underline{\textit{Proof}}: Let $x, y \in \mathbb{Z}$. Suppose $3 \nmid x$ and $3 \nmid y$. Here,
\begin{gather*}
x^2 - y^2 =(x+y)(x-y)
\end{gather*}
\underline{\textit{Case $1$}}: $\exists k, l \in \mathbb{Z},~s.t.~ x = 3k + 1 ~\mathrm{and}~ y = 3l + 1$.
Then,
\begin{gather*}
(x+y)(x-y) = (3k + 1 + 3l + 1)\{3k + 1 -(3l + 1)\} = 3(3(k+l)+2)(k-l)
\end{gather*}
Since $ (3(k+l)+2)(k-l) \in \mathbb{Z}, ~3 \mid (x^2 - y^2)$.\\[1em]
\underline{\textit{Case $2$}}: $\exists k, l \in \mathbb{Z},~s.t.~ x = 3k + 1 ~\mathrm{and}~ y = 3l + 2$.
Then,
\begin{gather*}
(x+y)(x-y) = (3k + 1 + 3l + 2)\{3k + 1 -(3l + 2)\} = 3(k+l+1)(3(k-l)-1)
\end{gather*}
Since $ (k+l+1)(3(k-l)-1) \in \mathbb{Z}, ~3 \mid (x^2 - y^2)$.\\[1em]
\underline{\textit{Case $3$}}: $\exists k, l \in \mathbb{Z},~s.t.~ x = 3k + 2 ~\mathrm{and}~ y = 3l + 1$.
Then,
\begin{gather*}
(x+y)(x-y) = (3k + 2 + 3l + 1)\{3k + 2 -(3l + 1)\} = 3(k+l+1)(3(k-l)+1)
\end{gather*}
Since $ (k+l+1)(3(k-l)+1) \in \mathbb{Z}, ~3 \mid (x^2 - y^2)$.\\[1em]
\underline{\textit{Case $4$}}: $\exists k, l \in \mathbb{Z},~s.t.~ x = 3k + 2 ~\mathrm{and}~ y = 3l + 2$.
Then,
\begin{gather*}
(x+y)(x-y) = (3k + 2 + 3l + 2)\{3k + 2 -(3l + 2)\} = 3(3(k+l+1)+1)(k-l)
\end{gather*}
Since $ (3(k+l+1)+1)(k-l) \in \mathbb{Z}, ~3 \mid (x^2 - y^2)$.\\[1em]
From the above, the statement is true.~$\blacksquare$

\section*{Problem.5}
\underline{\textit{Proof}}: Let $m, n \in \mathbb{N} ~s.t.~ m \mid n$.
Suppose $a, b \in \mathbb{Z}~s.t.~a \equiv b~(\mathrm{mod}~n)$. By definition,
\begin{gather*}
\exists k \in \mathbb{Z},~s.t.~ n = m \cdot k
\end{gather*}
and since $n \mid (b - a)$,
\begin{gather*}
\exists l \in \mathbb{Z},~s.t.~ b - a = n \cdot l
\end{gather*}
Thus, 
\begin{gather*}
b - a = m \cdot k \cdot l = m (kl)
\end{gather*}
where $kl \in \mathbb{Z}$. Therefore,
\begin{gather*}
n \mid (b - a) \Leftrightarrow a \equiv b~(\mathrm{mod}~m)~\blacksquare.
\end{gather*}

\section*{Problem.6}
\underline{\textit{Proof}}: Let $n \in \mathbb{Z}$.\\[1em]
($\Rightarrow$)\\[1em]
Suppose $ 2 \mid (n^4 - 3)$. By definition, $n^4-3 \in \mathbb{E}$. Therefore, $n^4 \in \mathbb{O}$ ($\because$ Theorem 3.16). Thus, $n^2 \in \mathbb{O}$, $n \in \mathbb{O}$ ($\because$ Corollary 1). By definition,
\begin{gather*}
\exists k \in \mathbb{Z},~s.t.~ n = 2k + 1
\end{gather*}
So,
\begin{gather*}
n^2 + 3 = (2k + 1)^2 + 3 = 4k^2 + 4k + 4 = 4(k^2 + k + 1)
\end{gather*}
Since $k^2 + k + 1 \in \mathbb{Z}$, $4 \mid n^2 + 3$.\\[1em]
($\Leftarrow$)\\[1em]
We prove by contrapositive. Suppose $ 2 \nmid (n^4 - 3)$. By definition, $n^4-3 \in \mathbb{O}$. Therefore, $n^4 \in \mathbb{E}$ ($\because$ Theorem 3.16). Thus, $n^2 \in \mathbb{E}$, $n \in \mathbb{E}$ ($\because$ Corollary 2). By definition,
\begin{gather*}
\exists k \in \mathbb{Z},~s.t.~ n = 2k
\end{gather*}
So,
\begin{gather*}
n^2 + 3 = (2k)^2 + 3 = 4(k^2) + 3
\end{gather*}
Since $k^2 \in \mathbb{Z}$, $4 \nmid n^2 + 3$.\\[1em]
From the above, the statement is true.~$\blacksquare$

\section*{Problem.7}
\underline{\textit{Proof}}: Let $a,b \in \mathbb{Z}$.\\[1em]
($\Leftarrow$)\\[1em]
\underline{\textit{Case $1$}}: $a \equiv b \equiv 0~(\mathrm{mod}~3)$.
By definition,
\begin{gather*}
\exists k, l \in \mathbb{Z},~s.t.~ a = 3k,~b = 3l
\end{gather*}
Therefore,
\begin{gather*}
a^2 + 2 b^2 - 0 = 9 k^2 + 18 l^2 = 3(3 k^2 + 6 l^2)
\end{gather*}
where $3 k^2 + 6 l^2 \in \mathbb{Z}$. Thus,
\begin{gather*}
3 \mid (a^2 + 2 b^2 - 0) \Leftrightarrow a^2 + 2 b^2 \equiv 0~(\mathrm{mod}~3).
\end{gather*}
\underline{\textit{Case $2$}}: $a \not \equiv 0, b \not \equiv 0 ~(\mathrm{mod}~3)$.\\[1em]
\underline{\textit{Subcase $2.1$}}: $a \equiv 1, b \equiv 1 ~(\mathrm{mod}~3)$.
By definition,
\begin{gather*}
\exists k, l \in \mathbb{Z},~s.t.~ a = 3k + 1,~b = 3l + 1.
\end{gather*}
Therefore,
\begin{align*}
a^2 + 2 b^2 - 0 &= (3k + 1)^2 + 2 (3l + 1)^2\\[1em]
&= 9k^2 + 6k + 1 + 2(9l^2 + 6l +1)\\[1em]
&= 3(3(k^2+2l^2) + 2(k+ 2l) + 1)
\end{align*}
where $3(k^2+2l^2) + 2(k+ 2l) + 1 \in \mathbb{Z}$. Thus,
\begin{gather*}
3 \mid (a^2 + 2 b^2 - 0) \Leftrightarrow a^2 + 2 b^2 \equiv 0~(\mathrm{mod}~3).
\end{gather*}
\underline{\textit{Subcase $2.2$}}: $a \equiv 1, b \equiv 2 ~(\mathrm{mod}~3)$.
By definition,
\begin{gather*}
\exists k, l \in \mathbb{Z},~s.t.~ a = 3k + 1,~b = 3l + 2.
\end{gather*}
Therefore,
\begin{align*}
a^2 + 2 b^2 - 0 &= (3k + 1)^2 + 2 (3l + 2)^2\\[1em]
&= 9k^2 + 6k + 1 + 2(9l^2 + 12l + 4)\\[1em]
&= 3(3(k^2+2l^2) + 2(k+ 4l) + 3)
\end{align*}
where $3(k^2+2l^2) + 2(k+ 4l) + 3 \in \mathbb{Z}$. Thus,
\begin{gather*}
3 \mid (a^2 + 2 b^2 - 0) \Leftrightarrow a^2 + 2 b^2 \equiv 0~(\mathrm{mod}~3).
\end{gather*}
\underline{\textit{Subcase $2.3$}}: $a \equiv 2, b \equiv 2 ~(\mathrm{mod}~3)$.
By definition,
\begin{gather*}
\exists k, l \in \mathbb{Z},~s.t.~ a = 3k + 2,~b = 3l + 2.
\end{gather*}
Therefore,
\begin{align*}
a^2 + 2 b^2 - 0 &= (3k + 2)^2 + 2 (3l + 2)^2\\[1em]
&= 9k^2 + 12k + 4 + 2(9l^2 + 12l + 4)\\[1em]
&= 3(3(k^2+2l^2) + 4(k+ 2l) + 4)
\end{align*}
where $3(k^2+2l^2) + 4(k+ 2l) + 4 \in \mathbb{Z}$. Thus,
\begin{gather*}
3 \mid (a^2 + 2 b^2 - 0) \Leftrightarrow a^2 + 2 b^2 \equiv 0~(\mathrm{mod}~3).
\end{gather*}
($\Rightarrow$)\\[1em]
Prove by contrapositive. 
Let the statements:
\begin{center}
$A$: $a \equiv 0$ and $b \equiv 0~(\mathrm{mod}~3)$\\[1em]
$B$: $a \not \equiv 0$ and $b \not \equiv 0 ~(\mathrm{mod}~3)$
\end{center}
and the compound statement:
\begin{gather*}
R \equiv A \lor B
\end{gather*}
Then
\begin{center}
$\sim R \equiv \sim(A \lor B) \equiv (\sim A) \land (\sim B)$ ~~($\because$ De Morgan's law) 
\end{center}
Therefore, suppose
\begin{gather*}
a \not \equiv 0~\mathrm{and}~b \equiv 0~(\mathrm{mod}~3),~\mathrm{or}, ~a \equiv 0~\mathrm{and}~b \not \equiv 0~(\mathrm{mod}~3)
\end{gather*}
\underline{\textit{Case $1$}}: $a \not \equiv 0~\mathrm{and}~b \equiv 0~(\mathrm{mod}~3)$.\\[1em]
\underline{\textit{Subcase $1.1$}}: $a \equiv 1~(\mathrm{mod}~3)$.
By definition,
\begin{gather*}
\exists k, l \in \mathbb{Z},~s.t.~ a = 3k + 1,~b = 3l.
\end{gather*}
Therefore,
\begin{align*}
a^2 + 2 b^2 - 0 &= (3k + 1)^2 + 2 (3l)^2\\[1em]
&= 9k^2 + 6k + 1 + 2(9l^2)\\[1em]
&= 3(3(k^2+2l^2) + 2k) + 1
\end{align*}
where $3(k^2+2l^2) + 2k \in \mathbb{Z}$. Thus,
\begin{gather*}
3 \nmid (a^2 + 2 b^2 - 0) \Leftrightarrow a^2 + 2 b^2 \not \equiv 0~(\mathrm{mod}~3).
\end{gather*}
\underline{\textit{Subcase $1.2$}}: $a \equiv 2~(\mathrm{mod}~3)$.
By definition,
\begin{gather*}
\exists k, l \in \mathbb{Z},~s.t.~ a = 3k + 2,~b = 3l.
\end{gather*}
Therefore,
\begin{align*}
a^2 + 2 b^2 - 0 &= (3k + 2)^2 + 2 (3l)^2\\[1em]
&= 9k^2 + 12k + 4 + 2(9l^2)\\[1em]
&= 3(3(k^2+2l^2) + 4k + 1) + 1
\end{align*}
where $3(k^2+2l^2) + 4k + 1 \in \mathbb{Z}$. Thus,
\begin{gather*}
3 \nmid (a^2 + 2 b^2 - 0) \Leftrightarrow a^2 + 2 b^2 \not \equiv 0~(\mathrm{mod}~3).
\end{gather*}
\underline{\textit{Case $2$}}: $a \equiv 0~\mathrm{and}~b \not \equiv 0~(\mathrm{mod}~3)$.\\[1em]
\underline{\textit{Subcase $2.1$}}: $b \equiv 1~(\mathrm{mod}~3)$.
By definition,
\begin{gather*}
\exists k, l \in \mathbb{Z},~s.t.~ a = 3k,~b = 3l + 1.
\end{gather*}
Therefore,
\begin{align*}
a^2 + 2 b^2 - 0 &= (3k)^2 + 2 (3l + 1)^2\\[1em]
&= 9k^2 + 2(9l^2 + 6l + 1)\\[1em]
&= 3(3(k^2+2l^2) + 4l) + 2
\end{align*}
where $3(k^2+2l^2) + 4l \in \mathbb{Z}$. Thus,
\begin{gather*}
3 \nmid (a^2 + 2 b^2 - 0) \Leftrightarrow a^2 + 2 b^2 \not \equiv 0~(\mathrm{mod}~3).
\end{gather*}
\underline{\textit{Subcase $2.2$}}: $b \equiv 2~(\mathrm{mod}~3)$.
By definition,
\begin{gather*}
\exists k, l \in \mathbb{Z},~s.t.~ a = 3k,~b = 3l + 2.
\end{gather*}
Therefore,
\begin{align*}
a^2 + 2 b^2 - 0 &= (3k)^2 + 2 (3l + 2)^2\\[1em]
&= 9k^2 + 2(9l^2 + 12l + 4)\\[1em]
&= 3(3(k^2+2l^2) + 8l + 2) + 2
\end{align*}
where $3(k^2+2l^2) + 8l + 2 \in \mathbb{Z}$. Thus,
\begin{gather*}
3 \nmid (a^2 + 2 b^2 - 0) \Leftrightarrow a^2 + 2 b^2 \not \equiv 0~(\mathrm{mod}~3).
\end{gather*}
From the above, the statement is true.~$\blacksquare$
\end{document}
























