\documentclass[12pt]{article}
\usepackage{fullpage}
\usepackage{graphicx}
\usepackage{hyperref}
\usepackage{bm}
\usepackage{amsmath}
\usepackage{amssymb}
\usepackage{derivative}
\usepackage{bm}
\usepackage{comment}
\usepackage{cancel}
\usepackage{xcolor}


\begin{document}
\title{Mathematical Proof: Problem Set 5}
\author{Koichiro Takahashi}
\maketitle 
\noindent We know the triangle inequality from the textbook(or our class):
\begin{center}
Theorem 4.17:
Let $a,b \in \mathbb{R}$. $|a+b| \leq |a| + |b|$.
\end{center}

\section*{Problem.1}
\underline{\textit{Proof}}: Let $a, b, c, d \in \mathbb{R}$.
Here,
\begin{align*}
(ac+bd)^2 &= a^2 c^2 + b^2 d^2 + 2abcd\\[1em]
(ab-cd)^2 &= a^2 b^2 + c^2 d^2 - 2abcd
\end{align*}
Since the square of any real number is non-negative, observe
\begin{gather*}
(ab - cd)^2 \geq 0
\end{gather*}
$\Leftrightarrow$
\begin{gather*}
a^2 b^2 + c^2 d^2 - 2abcd \geq 0
\end{gather*}
$\Leftrightarrow$
\begin{gather*}
2abcd \leq a^2 b^2 + c^2 d^2
\end{gather*}
By adding the same terms on both hand sides,
\begin{gather*}
a^2 c^2 + b^2 d^2 + 2abcd \leq a^2 b^2 + a^2 c^2 + b^2 d^2 + c^2 d^2
\end{gather*}
$\Leftrightarrow$
\begin{gather*}
(ac+bd)^2 \leq (a^2 + c^2)(b^2 + d^2)
\end{gather*}
$\Rightarrow$
\begin{gather*}
ac+bd \leq |ac+bd| \leq \sqrt{(a^2 + c^2)(b^2 + d^2)} = \sqrt{(a^2 + c^2)} \cdot \sqrt{(b^2 + d^2)}
\end{gather*}
From the above, the statement is true.~$\blacksquare$

\section*{Problem.2}
\underline{\textit{Proof}}: Let $x, y, z \in \mathbb{Z}$.
Here, $(x-y) + (y-z) = x-z$, also $x-z,x-y,y-z \in \mathbb{R}$. By applying the triangle inequality above as $a = x-y,~b = y-z$, we get
\begin{gather*}
|x-z| \leq |x-y| + |y-z|. ~\blacksquare
\end{gather*}

\section*{Problem.3}
\underline{\textit{Proof}}: Let $a, b \in \mathbb{R} ~s.t.~ a > 0,~b > 0$.\\[1em]
Since the square of any real number is non-negative, observe
\begin{gather*}
(a-b)^2 = a^2 + b^2 - 2ab \geq 0
\end{gather*}
$\Leftrightarrow$
\begin{gather*}
a^2 + b^2 \geq 2ab
\end{gather*}
$\Leftrightarrow$
\begin{gather*}
\frac{a^2 + b^2}{ab} \geq 2 ~~(\because ab > 0)
\end{gather*}
$\Leftrightarrow$
\begin{gather*}
\frac{a}{b} + \frac{b}{a} \geq 2
\end{gather*}
From the above, the statement is true.~$\blacksquare$\\[1em]
By following the proof above backwards as a equation, immediately we see that
\begin{gather*}
\frac{a}{b} + \frac{b}{a} = 2 \Leftrightarrow (a-b)^2 = 0 \Leftrightarrow a = b
\end{gather*}
Therefore, the complete solution set $U = \{(a,b)~|~ \forall (a, b) \in \mathbb{R}^2 ~s.t.~ a>0,~b>0,~a=b\}$.


\section*{Problem.4}
\underline{\textit{Proof}}: Let $A$ and $B$ be sets.\\[1em]
($\Leftarrow$)\\[1em]
Suppose $A = B$, by definition, $\forall x \in A,~ x \in B$ and $\forall x \in B,~ x \in A$.\\
We prove $(A \cup B) \subseteq (A \cap B)$ and $(A \cup B) \supseteq (A \cap B)$.\\[1em]
($\subseteq$) Let $x \in (A \cup B)$.\\
\underline{\textit{Case $1$}}: $x \in A$. Then $x \in B ~(\because A = B)$. Therefore, $x \in (A \cap B)$.\\
\underline{\textit{Case $2$}}: $x \in B$. Then $x \in A ~(\because A = B)$. Therefore, $x \in (A \cap B)$.\\
Thus, $(A \cup B) \subseteq (A \cap B)$.\\[1em]
($\supseteq$) Let $x \in (A \cap B)$. By definition, $x \in A$ and $x \in B$. Therefore, $x \in (A \cup B)$.\\
Thus, $(A \cup B) \supseteq (A \cap B)$.\\[1em]
So, $(A \cup B) = (A \cap B)$.\\[1em]
($\Rightarrow$)\\[1em]
Suppose $(A \cup B) = (A \cap B)$. We prove $A \subseteq B$ and $A \supseteq B$.\\[1em]
($\subseteq$) $\forall x \in A \subseteq (A \cup B) = (A \cap B)$. Therefore, $x \in B$, $A \subseteq B$.\\[1em]
($\supseteq$) $\forall x \in B \subseteq (A \cup B) = (A \cap B)$. Therefore, $x \in A$, $A \supseteq B$.\\[1em]
So, $A = B$.\\[1em]
From the above, the statement is true.~$\blacksquare$

\section*{Problem.5}
We prove that 
\begin{gather*}
(A \times B) \cap (B \times A) = \emptyset \Leftrightarrow  A \cap B = \emptyset
\end{gather*}
\underline{\textit{Proof}}: Let $A$ and $B$ be sets.\\[1em]
($\Leftarrow$)\\[1em]
Suppose $A \cap B = \emptyset$, by definition, $\forall x \in A$, $x \notin B$, and $\forall y \in B$, $y \notin A$.\\
Let $a \in (A \times B)$, then $\exists x \in A, \exists y \in B,~ s.t.~ a = (x,y)$, but $a \notin (B \times A)$, since $x \notin B$, and $y \notin A$.\\ Similarly, let $b \in (B \times A)$, then $\exists y \in B, \exists x \in A,~ s.t.~ b = (y,x)$, but $b \notin (A \times B)$, since $y \notin A$, and $x \notin B$.\\
Therefore, $(A \times B) \cap (B \times A) = \emptyset$.\\[1em]
($\Rightarrow$)\\[1em]
Suppose $(A \times B) \cap (B \times A) = \emptyset$. Let $x \in A, y \in B$, then $\exists a \in (A \times B) ~s.t.~ a = (x,y)$, and also, $\exists b \in (B \times A) ~s.t.~ b = (y, x)$. However, since $(A \times B) \cap (B \times A) = \emptyset$, $a \neq b \Rightarrow x \neq y$. So $x \notin B, y \notin A$, $A \cap B = \emptyset$.\\[1em]
From the above, the statement is true.~$\blacksquare$

\section*{Problem.6}
\underline{\textit{Proof}}: Let $A,~B,~C$ and $D$ be sets.\\[1em]
We prove $(A \times B) \cap (C \times D) \subseteq (A \cap C) \times (B \cap D)$ and $(A \times B) \cap (C \times D) \supseteq (A \cap C) \times (B \cap D)$.\\[1em]
($\subseteq$)\\[1em]
Let $w \in (A \times B) \cap (C \times D)$, by definition, $w \in (A \times B)$, and $w \in (C \times D)$. By definition, $\exists x \in A, \exists y \in B ~s.t.~ w = (x,y)$, but also $x \in C, y \in D ~(\because w \in (C \times D))$. Therefore, $x \in A$ and $x \in C$, also $y \in B$ and $y \in D$. Thus, $w = (x, y) \in (A \cap C) \times (B \cap D)$, $(A \times B) \cap (C \times D) \subseteq (A \cap C) \times (B \cap D)$.\\[1em]
($\supseteq$)\\[1em]
Let $w \in (A \cap C) \times (B \cap D)$, by definition, $\exists x \in (A \cap C), \exists y \in (B \cap D) ~s.t.~ w = (x,y)$. By definition, $x \in A$ and $x \in C$, also $y \in B$ and $y \in D$. Therefore, $w = (x, y) \in (A \times B)$ and $w = (x, y) \in (C \times D)$. Thus, $w \in (A \times B) \cap (C \times D)$, $(A \times B) \cap (C \times D) \supseteq (A \cap C) \times (B \cap D)$.\\[1em]
From the above, the statement is true.~$\blacksquare$

\end{document}
























